\documentclass[]{article}
\usepackage{amsmath}
\usepackage{amsfonts}
\usepackage{amssymb}
\usepackage{hyperref}
\usepackage{gensymb}
\usepackage{graphicx}
\usepackage{svg}
\usepackage{bbding}
\usepackage{mathtools}
\usepackage{centernot} % not parallel, etc.
\usepackage{lmodern}
\usepackage{morewrites}
\usepackage{xcolor,sectsty} % colorful sections
\usepackage[left=10mm, top=10mm, right=10mm, bottom=20mm, nohead]{geometry}
%\usepackage{bigints}
\usepackage{dsfont} %mathbb 1
\usepackage{esint} % beatiful integrals
\usepackage[arrowdel]{physics}
\usepackage{amsthm} % theorems


\usepackage[T1]{fontenc}
% Nicer default font (+ math font) than Computer Modern for most use cases
% \usepackage{mathpazo} % problems with greek vectors
\usepackage[utf8x]{inputenc} % Allow utf-8 characters in the tex document
% Prevent overflowing lines due to hard-to-break entities
\sloppy 
% Colors for the hyperref package
\definecolor{urlcolor}{rgb}{0,.145,.698}
\definecolor{linkcolor}{rgb}{.71,0.21,0.01}
\definecolor{citecolor}{rgb}{.12,.54,.11}
% Setup hyperref package
\hypersetup{
	breaklinks=true,  % so long urls are correctly broken across lines
	colorlinks=true,
	urlcolor=urlcolor,
	linkcolor=linkcolor,
	citecolor=citecolor,
}


\DeclareFontFamily{OMX}{lmex}{}
\DeclareFontShape{OMX}{lmex}{m}{n}{<-> lmex10}{}


%colors of sections
\definecolor{secfont}{RGB}{46,116,181}
\definecolor{subfont}{RGB}{146,23,57}
\definecolor{parfont}{RGB}{19,127,43}
\definecolor{subparfont}{RGB}{7,11,100}

\subsectionfont{\color{subfont}}
\sectionfont{\color{secfont}}
\paragraphfont{\color{parfont}}
\subparagraphfont{\color{subparfont}}

%\usepackage{babel}[english]
%opening
\title{116217 - Solid State Physics}
\author{Netanel Lindner}
% % Simon, Oxford Solid State PhySics
% Ashcroft and Merim
% Kittel

\newtheorem{theorem}{Theorem}[section]
\newtheorem{definition}{Definition}[section]
\newtheorem{coll}{Collary}[section]
\newtheorem{lemma}{Lemma}[section]

\parindent=0em
\begin{document}


\maketitle

\begin{abstract}

\end{abstract}

%\tableofcontents
\section{Introduction}
Phase are distinguished by symmetries, e.g., liquid and solid have difference in transnational symmetry. Those phase transitions happens together with spontaneous symmetry breakage.For example, magnet ''chooses`` poles when transits to magnetic phase.
\subsection{Heat capacity}
Most of matereals have $C=3k_B$, except diamond in room temperature and pressure. In low temperatures though, materials have $C=\alpha T + \gamma T^3$. Thus, most of materials get to saturation in room temperature, while diamond doesn't.
The linear factor appears only in conductors, thus should come from electrons.
\paragraph{Equipartition theorem}
$$
\expval{x_{m} \frac{\partial H}{\partial x_{n}}}  = \delta_{mn} k_{B} T
$$
in particular, for quadratic degree of freedom we get $\frac{1}{2}k_BT$.

Thus, assuming each atom is harmonic oscillator, we have
$$\expval{E} = 3k_BT$$
and
$$c_V = 3k_B$$
\paragraph{Einstein model}
Now assume the oscillators are quantum:
$$E_n=\hbar \omega\qty(n+\frac{1}{2})$$
The partition function
$$Z = \sum_n e^{-\beta E_n} = \sum_n e^{-\beta \hbar \omega \qty(n+\frac{1}{2})} = e^{-\frac{\beta \hbar \omega}{2}}\frac{1}{1-e^{-\beta \hbar \omega}} = \frac{e^{-\frac{\beta \hbar \omega}{2}}}{e^{-\frac{\beta \hbar \omega}{2}} \qty[e^{\frac{\beta \hbar \omega}{2}}-e^{-\frac{\beta \hbar \omega}{2}}]} = \frac{1}{2\sinh(\frac{\beta \hbar \omega}{2})}$$
$$\expval{E}  = -\pdv{\beta} \ln Z_N = -\frac{1}{Z_N} \pdv{\beta} Z = N \cdot 2\sinh(\frac{\beta \hbar \omega}{2}) \cdot \frac{\hbar \omega}{2}\frac{\cosh(\frac{\beta \hbar \omega}{2})}{\sinh[2](\frac{\beta \hbar \omega}{2})} = \frac{N\hbar \omega}{2} \frac{1}{\tanh(\frac{\beta \hbar \omega}{2})}$$
$$\expval{E}  = N\hbar\omega \qty(n_B(\beta \hbar \omega)+\frac{1}{2})$$
$$n_B(x) = \frac{1}{r^x-1}$$

Now
$$c_V = \pdv{\expval{E}}{T} = 3N k_B (\beta \hbar  \omega)^2 \frac{e^{\beta \hbar \omega}}{\qty(e^{\beta \hbar \omega}-1)^2}$$
$$\lim_{T\to \infty } c_V = 3k_BN$$
$$\lim_{T\to 0 } c_V = 0$$
which is expected, however, we got exponential dependence on $T$ and not $T^3$.
\subsection{Debye model}
Debye proposed to model atoms in solid as sound waves. Plank already quantized EM-waves, the only differences is number of polarization options and speed of wave.


We'll often use periodic bound conditions, since this preserves the symmetry of the system.

Now, for wave $e^{ikx}$ we get $e^{ikx}=e^{ik(x+L)}$, thus we get a condition on wavenumber
$$k=\frac{2\pi}{L}n$$

We also want to switch to integral on $k$.

The connection between frequency and wavenumber is $\omega(\va{k}) = v\cdot \abs{\va{k}}$. The energy per wavenumber
$$\expval{E} = 3\sum_{\va{k}} \hbar \omega(\va{k}) \qty(n_B(\beta\hbar\omega(\va{k}))+\frac{1}{2}) $$
$$\expval{E} = 3\frac{L^3}{(2\pi)^3} \int \dd[3]{k} \hbar \omega(\va{k}) \qty(n_B(\beta\hbar\omega(\va{k}))+\frac{1}{2}) =3 \frac{4\pi L^3}{(2\pi)^3} \int \frac{\omega^2 \dd{\omega}}{v^3} \hbar \omega  \qty(n_B(\beta\hbar\omega)+\frac{1}{2})$$

Now,
$$g(\omega) = L^3 \cdot \frac{12\pi \omega^2}{(2\pi)^3 v^3} = N \frac{12\pi \omega^2}{(2\pi)^3 v^3\rho}$$
is density of states.

Define $\omega_d^3 = 6\pi^2 \rho v^3$, acquiring $g(\omega) = N\frac{9\omega^2}{\omega_d^3} $
$$\expval{E} = \frac{9N\hbar }{\omega_d^3} \int \omega^3 \dd{\omega} \frac{1}{e^{\beta \hbar\omega }-1} + \text{const}$$

Substituting $\beta \hbar \omega = x$:
$$\expval{E} = \frac{9N\hbar }{\omega_d^3 (\beta \hbar)^4} \underbrace{\int \dd{x} x^3 \frac{1}{e^x-1}}_{\frac{\pi^4}{15}} $$

$$\pdv{\expval{E}}{T} = \frac{12\pi^4}{5}\frac{nk_B \qty(k_BT)^3}{(\hbar \omega_d)^3} $$
The reason we get divergence in high temperatures is that we assume any frequency is possible, however, the wavelength can't be less than distance between atoms.

The total number of possible frequencies equals to number of degrees of freedom:
$$\int_0^{\omega_{cutoff}} \dd{\omega} g(\omega) = 3N$$
$$\int_0^{\omega_{cutoff}} \dd{\omega}N \frac{9\omega^2}{\omega_d^3} = N\cdot \frac{3\omega^3}{\omega_d^3}=3N$$
i.e., $\omega_{cutoff} = \omega_d$:
$$\expval{E} = \int_0^{\omega_d} \frac{\dd{\omega} g(\omega) \omega^3}{e^{\beta \hbar \omega}-1}$$

In low temperatures our calculation is yet good, in high temperatures, using Taylor expansion of $e^{\beta \hbar \omega}$
$$\expval{E} = \int_0^{\omega_d} \frac{\dd{\omega} g(\omega) \omega^3}{\beta \hbar \omega} = \int_0^{\omega_d} ]\dd{\omega} g(\omega) k_B T = 3Nk_B T$$

We define some different qunatities based on Debye frequency
$T_d = \frac{\hbar \omega_d}{k_B}$
$k_d = \frac{ \omega_d}{v}$

\subsection{Phonons}
Suppose we have a one-dimensional chain of atoms of mass $m$ in position
$$x_n^{eq} = na$$
with  oscillations
$$u =x_n - x_n^{eq}$$
If $u$ is small it can be approximated with quadratic potential, i.e., a chain of springs.

$$V_{tot} = \sum_j V(x_{j+1}-x_j) = \sum_j \frac{K}{2} \qty(x_{j+1} -x_j -a)^2 = \sum_j \frac{K}{2} (u_{j+1}-u_j)^2 $$
$$F_n = -\pdv{V_{tot}}{u_n} = K(u_{n+1}-u_n) + K(u_{n-1}-u_n)$$

We get system of equations
$$m\ddot{u}_n = F_n = K(u_{n+1} + u_{n-1} - 2u_n)$$
Or, in matrix form
$$m\begin{pmatrix}
\ddot{u}_1\\\ddot{u}_2\\\vdots \\\ddot{u}_n
\end{pmatrix} = K\begin{pmatrix}
-2&1&0&\dots&0\\
1&-2&1&\dots& 0\\
\vdots&\vdots&\vdots&\ddots&\vdots\\
0&0&0&\dots &-2
\end{pmatrix}\begin{pmatrix}
u_1\\u_2\\\vdots \\u_n
\end{pmatrix} $$

Guessing a solution
$$u_n = A e^{i\omega t} e^{-ikan}$$
We get
$$-m\omega^2u_n = K A  e^{i\omega t}\qty( e^{-ika(n+1)}+e^{-ika(n-1)} - 2e^{-ikan} ) = Ku_n(e^{-ika} + e^{ika} -2)$$
$$-m\omega^2 = 2K\qty(\cos(ak)-1)$$
i.e.,
$$m\omega^2 = 4K \sin[2](\frac{ka}{2})$$
or,
$$\omega(k) = 2\sqrt{\frac{K}{m}} \abs{\sin(\frac{ka}{2})}$$
The dispersion has minimum and maximum. We can interval of allowed frequencies ''band`` and its size is called ''bandwidth``. 
The dispersion is periodic with period of $\frac{2\pi}{a}$, since the system is symmetric to translations of $a$.

\paragraph{Solutions of the equations}
Since translation operator commutes with Hamiltonian, the eigenfunctions of Hamiltonian got to be eigenfunctions of the translation operator, which are
$$u_N = A e^{i\omega t - ikna}$$
\subsection{Reciprocal lattice}
\paragraph{Reciprocal space}
Space of wavenumbers. We found equivalent points in inverse space. We call all the points equivalent to $k=0$ reciprocal lattice, i.e., $k=\frac{2\pi}{a} p$ for $p \in \mathbb{Z}$.
\paragraph{First Brillouin zone}
is a unit cell centered in $k=0$.
\paragraph{Finite system}
If the system is finite with size $L$,
$$e^{ik na} = e^{ik \qty(n+\frac{L}{a})a}$$
Thus
$$k=\frac{2\pi}{L} \cdot p$$

i.e., the number of modes
$$N_{modes} = \frac{\frac{2\pi}{a}}{\frac{2\pi}{L}} = \frac{L}{a} = N_{atoms}$$

\paragraph{Properties of dispersion}
\subparagraph{Sound waves}
$$\omega(k) \approx \sqrt{\frac{K}{m}} \abs{k}a = v\abs{k}$$
\subparagraph{Short wavelengths}
$$k \approx \frac{\pi}{a}$$
\subparagraph{Velocity}
$$v_{group} = \dv{\omega}{k}$$
$$v_{phase} = \frac{\omega}{k}$$

\subsection{Quantized phonons }
We quantize our oscillator , with energy
$$E_N =\hbar \omega \qty(n+\frac{1}{2})$$
\paragraph{Crystal momentum}
We define crystal momentum, momentum modulo $\hbar\frac{2\pi}{a}$. Since the lattice can give or take momentum in portions $\hbar\frac{2\pi}{a}$.
\subsection{Two-atom chain of atoms}
Suppose we have two types of atoms, $x$ and $y$, with locations
$$x_n  = na+\frac{a}{10}$$
$$y_n  = na+\frac{8a}{10}$$
Suppose $m_1=m_2$ but $k_1\neq k_2$, ($_2$ connects $x_n$ with $y_n$ and $k_2$ connects $y_n$ with $x_{n+1}$).
$$m\delta \ddot{x}_n = k_2(\delta y_n - \delta x_n) + k_1(\delta y_{n-1} - \delta x_n)$$
$$m\delta \ddot{y}_n =k_1(\delta x_{n+1} - \delta y_{n}) + k_2(\delta x_n - \delta y_n ) $$

Substitute $\delta x_n =A_x e^{i\omega t - ikna}$ and $\delta y_n =A_y e^{i\omega t - ikna}$:
$$\begin{cases}
-\omega^2 A_x m  e^{i\omega t - ikna} = k_2\qty(A_y e^{i\omega t - ikna} - A_x e^{i\omega t - ikna}) + k_1\qty(A_y e^{i\omega t - ik(n-1)a} - A_x e^{i\omega t - ikna})\\-\omega^2 A_y m  e^{i\omega t - ikna}  =k_1\qty(A_x e^{i\omega t - ik(n+1)a} - A_y e^{i\omega t - ikna}) + k_2\qty(A_x e^{i\omega t - ikna} - A_y e^{i\omega t - ikna})
\end{cases} $$

$$\begin{cases}
-\omega^2  mA_x = (e^{ika} k_1 + k_2)A_y  - (k_1+k_2)A_x \\
-\omega^2  m A_y   =\qty(k_1 e^{ - ika} +k_2) A_x -  (k_1+k_2)A_y
\end{cases} $$

In matrix form
$$m\omega^2 \begin{pmatrix}A_x\\A_y\end{pmatrix} = \begin{pmatrix}
- (k_1+k_2)&e^{ika} k_1 + k_2\\
k_1 e^{ - ika} +k_2 & -  (k_1+k_2)
\end{pmatrix} \begin{pmatrix}A_x\\A_y\end{pmatrix} $$
$$\begin{pmatrix}
m\omega^2 - (k_1+k_2)&-\qty(e^{ika} k_1 + k_2)\\
-\qty(k_1 e^{ - ika} +k_2 )& m\omega^2 -  (k_1+k_2)
\end{pmatrix} \begin{pmatrix}A_x\\A_y\end{pmatrix} $$

$$\qty[m\omega^2 - (k_1+k_2)]^2 + \abs{k_1 e^{ - ika} +k_2}^2 = 0$$
$$m\omega^2 = k_1+k_2 \pm \abs{k_1e^{ika} + k_2}$$
$$\omega = \sqrt{\frac{k_1+k_2}{m} \pm \frac{1}{m}\sqrt{k_1^2 + k_2^2+2k_1k_2\cos(ka)}} = \sqrt{\frac{k_1+k_2}{m} \pm \frac{1}{m}\sqrt{(k_1 + k_2)^2-4k_1k_2\sin[2](\frac{ka}{2})}}$$

Take a look at eigenvectors in $k=0$:
they are $\begin{pmatrix}
1\\1
\end{pmatrix}$ for eigenvalue $0$ and $\begin{pmatrix}
1\\-1
\end{pmatrix}$ for eigenvalue $2(k_1+k_2)$. 

\input{lect5.tex}
If we take $k_1\to k_2$,the gap starts to close. Since at $k_1=k_2$ the actual Brillouin zone is twice as big and the dispersion relation is $\sine[2]$, ta gap closes exactly at $k_1=k_2$.

\section{Drude model}
Model of conductance. Has three assumptions:
\begin{enumerate}
	\item Characteristic time of collision is $\tau$, i.e., probability of collision in time $\dd{t}$ is $\frac{\dd{t}}{\tau}$.
	\item After the collision average momentum is zero: $\expval{\va{p}} = 0$.
	\item Between collisions electron moves as usual.
\end{enumerate}

Lets write the equation
$$\expval{\va{p}(t+\dd{t}) } = 0\frac{\dd{t}}{\tau} + \qty(\expval{\va{p}(t)} + \va{F}\dd{t})\qty(1-\frac{\dd{t}}{\tau})$$
$$\dv{\expval{\va{p}(t)}}{t} = -\frac{\expval{\va{p}(t)}}{\tau} +\va{F}$$

If $\va{F} = 0$ we get exponential.

We want to study Lorenz force
$$\va{F}= -e\va{E} - e\qty(\va{v} \cross \va{B})$$

If we have only electric field ($\va{B}=0$)

$$\dv{\expval{\va{p}(t)}}{t} = -\frac{\expval{\va{p}(t)}}{\tau} - e\va{E}$$

We are interested in steady state, i.e., $\dv{\expval{\va{p}(t)}}{t} = 0$:
$$\frac{m\expval{\va{v}(t)}}{\tau} = - e\va{E}$$

Now we want to find current density:
 $$\va{j} = -en\va{v} = \frac{e^2 \tau n}{m} \va{v}$$
 
 We got Ohm law,
 $$\va{j} = \sigma \va{E}$$
 where $\sigma$ is conductivity
 $$\sigma = \frac{e^2n\tau}{m}$$
 
 \subsection{Hall effect}
 $$\dv{\va{p}}{t} = -e\qty(\va{E} + \va{v} \cross \va{B}) -\frac{\va{p}}{\tau}$$
 $$\dv{\va{p}}{t} = -e\va{E} +\frac{ \va{j} \cross \va{B} }{n}+\frac{m}{ne\tau} \va{j}$$
 
 We can write down $\va{E} = \rho \va{j}$ for some matrix $\rho$. Assume $\va{B} = B\vu{z}$.
 
 Then we get sum of antisymmetric matrix from cross product and scalar. $\rho$ is called resistivity tensor. We also define conductivity tensor $\sigma = \rho^{-1}$.
 
 Now suppose we measure current as function of $E_x$L
 $$\begin{cases}
 j_x = \sigma_{xx} E_x + \sigma_{xy} E_y\\
 j_y = \sigma_{yx} E_x + \sigma_{yy} E_y
 \end{cases}$$
 
 Since $j_y=0$,
 $$\sigma_{yx} E_x + \sigma_{yy} E_y = 0$$
 $$E_y = -\frac{\sigma_{yx}}{\sigma_{yy}} E_x = -\omega_c \tau E_x$$
 where $\omega_c = \frac{eB}{m}$ and thus
 $$j_x = \sigma_{xx} E_x + \sigma_{xy} E_y = \frac{\sigma_0}{1+(\omega_c \tau)^2} \qty(E_x - \omega_c \tau E_y) = \sigma_0 E_x$$
% TODO
\section{Sommerfeld theory}
We'll try to model electricity with quantum physics. Since electrons are fermions, we'll use Fermi-Dirac statistics.

We want to work in grand canonical ensemble. For that we replace Hamiltonian with $H-\nu \hat{N}$.

The partition function of grand canonical ensemble is
$$Z = \sum_{\alpha} e^{-\beta \qty(E_\alpha - \mu N_\alpha)}$$
The probability of finding particle in some state
$$P = \frac{e^{-\beta(E_h-\mu)}}{1+e^{-\beta(E_h-\mu)}}$$

Number of particles in particular energy:
$$n_F(\beta(E-\mu)) = \frac{1}{1+e^{\beta(E-\mu)}}$$


\paragraph{Fermi energy}
Suppose we have free particles in a box $L\times L \times L$ with periodic boundary conditions.


$$H= \frac{p^2}{2m}$$

The eigenfunctions are 
$$\psi_k = \frac{1}{\sqrt{v}}e^{i\va{k} \vdot \va{r}}$$

The energy of this state
$$\epsilon_k  = \frac{\hbar^2 \va{k}^2}{2m}$$

Total number of particles is
$$\expval{N} = 2\sum_{\va{k}} n_F(\beta(E-\mu)) $$
$$\expval{N} = 2\frac{V}{(2\pi)^3}\int \dd[3]{k} n_F(\beta(E-\mu)) $$

This equation defines $\mu$ (if we know number of particles).

\paragraph{Fermi energy} is $\mu$ at $T=0$, we denote it $\epsilon_F$. Fermi sea are all energy levels for which $\epsilon<\epsilon_F$.

We define Fermi momentum as maximal populated momentum. In this case
$$\epsilon_F = \frac{\hbar^2 k_F^2}{2m}$$
and Fermi velocity
$$v_F = \frac{\hbar k_F}{m}$$

\paragraph{Value of $\epsilon_F$ in 3D}
$$N = 2 \frac{V}{(2\pi)^3}\int \dd[3]{k} \Theta(\epsilon_F-\epsilon_k)$$
$$ 2 \frac{V}{(2\pi)^3} =\frac{4\pi}{3} k_F^3$$
$$n=\frac{N}{V} = \frac{1}{3\pi^2}k_F^3$$
$$k_F = (3\pi^2 n)^{\frac{1}{3}}$$
$$\epsilon_F = \frac{\hbar^2(3\pi^2 n)^{\frac{2}{3}}}{2m}$$


We also define Fermi temperature
$$T_F = \frac{\epsilon_F}{k_B}$$

\paragraph{Radius Bohr} $a_b = \frac{\hbar^2}{me^2} = 0.52 \AA$

$$k_F = \qty(\frac{9\pi}{n})^{\frac{1}{3}} \frac{1}{r_s}$$
$$k_F = \frac{3.63}{\frac{r_s}{a_r}}\circ{A}^{-1}$$

$$\frac{r_s}{a_r} \sim 1$$

\paragraph{Energy density}
$$\int \dd[3]{k} \leftrightarrow \int g(\epsilon) \dd{\epsilon}$$

$$g(\epsilon) \dd{\epsilon} = 4\pi k^2 \dd{k} \frac{2}{(2\pi)^3}$$
$$\dv{\epsilon}{k} = \frac{\hbar^2 k}{m}$$
$$g(\epsilon) = \frac{mk}{\pi^2 \hbar^2} = \frac{\sqrt{2} m^{\frac{3}{2}}}{\pi^2\hbar^3} \epsilon^{\frac{1}{2}}$$

\paragraph{Connection between chemical potential to dencity}
$$n = \int_0^\infty \dd{\epsilon} g(\epsilon ) n_F(\beta\qty(\epsilon-\mu)) = \frac{1}{(2\pi)^3} \int \dd[3]{k} n_F(\beta\qty(\epsilon_k-\mu))$$

$$n = \int_0^\infty \dd{\epsilon} \frac{\sqrt{2} m^{\frac{3}{2}}}{\pi^2\hbar^3} \frac{\epsilon^{\frac{1}{2}}}{e^{\beta(\epsilon-\mu) } +1}$$

\paragraph{Heat capacity}
Since 
$$E_{tot} = V\int_0^\infty g(\epsilon) \epsilon n_F(\beta(\epsilon-\mu))$$

We want to approximate $E$ when $T\ll T_F$. Note that if we count new electrons which got  energies higher than $\mu$:
$$n = \int_\mu^{\infty} \dd{\epsilon} g(\epsilon) n_F(\beta(\epsilon-\mu))$$
and those that disappeared from low energies

$$n = \int_0^{\mu} \dd{\epsilon} g(\epsilon) \qty(1-n_F(\beta(\epsilon-\mu)))$$

If $g(\epsilon)$ was constant, those would be equal. That would mean those electrons just moved to higher energies, each acquiring $k_BT$ energy (size of energy window),we get  (assuming $\mu$ would be constant) 
$$E(T) = E(T=0) + \frac{\gamma}{2} (V g(\epsilon_F) k_BT)k_BT$$
And thus
$$C_V = v\gamma  g(\epsilon) k^2 T$$
We are interested in integral
$$I = \int_{-\infty}^{\infty} \dd{\epsilon} H(\epsilon) n_F(\epsilon) $$

Define
$$K(\epsilon) = \int_{-\infty}^{\epsilon} \dd{\epsilon'} H(\epsilon') $$

Integrating by parts:
$$I = \underbrace{\eval{K(\epsilon ) n_F(\epsilon)}_{-\infty}^{\infty}}_{0} +\int_{-\infty}^\infty K(\epsilon) \qty(-\pdv{n_F}{\epsilon}) $$

Now
$$-\pdv{n_F}{\epsilon} = \frac{\beta e^{\beta\qty(\epsilon-\mu)}}{\qty( e^{\beta\qty(\epsilon-\mu)}+1)^2} = \frac{\beta}{4\cosh(\frac{\beta(\epsilon-\mu)}{2})}$$
Note that
$$\int_{-\infty}^\infty \dd{\epsilon} \qty(-\pdv{n_F}{\epsilon}) =1$$

Now we want to take Taylor series of $K$:

$$K(\epsilon) = K(\mu) + \sum_n \frac{(\epsilon-\mu)^n}{n!} \dv[n]{k}{\epsilon} \qty(\mu)$$

i.e.,
$$I = K(\mu)\int_{-\infty}^\infty \dd{\epsilon} \qty(\pdv{n_F}{\epsilon}) + \sum_n \int_{-\infty}^\infty \dd{\epsilon}  \frac{(\epsilon-\mu)^{2n}}{(2n)!} \qty[-\pdv{n_F}{\epsilon}] \dv[2n]{K}{\epsilon} \qty(\mu)$$

Define $x=\beta(\epsilon-\mu)$:
$$\int_{-\infty}^\infty \dd{\epsilon}  \frac{(\epsilon-\mu)^{2n}}{(2n)!} \qty[-\pdv{n_F}{\epsilon}] \dv[2n]{K}{\epsilon} \qty(\mu) = (k_BT)^{2n}\int_{-\infty}^\infty \dd{x} r_n(x) \dv[2n]{K}{\epsilon} \qty(\mu)$$

Define
$$a_n = \int_{-\infty}^\infty \dd{x} r_n(x) = \int_{-\infty}^\infty \dd{x} \frac{x^{2n}}{(2n)!} \qty(-\dv{x} \frac{1}{e^x+1})$$
We can get
$$a_n = \qty(2-\frac{1}{2^{2(n+1)}}) \zeta(2n)$$


Thus
$$I = \int_{-\infty}^\mu H(\epsilon) \dd{\epsilon} + \frac{\pi^2}{6} (kT)^2 H'(\mu) + \frac{7\pi^4}{360} (k_BT)^4 H^{(3)} (\mu) + \order{(k_BT)^6} $$

\paragraph{Chemical potential in low temperatures}
$$n = \int_{-\infty}^\infty g(\epsilon) n_F(\beta(\epsilon-\mu))$$
$$n = \underbrace{\int_{-\infty}^\mu g(\epsilon) \dd{\epsilon}}_{n} + \frac{\pi^2}{6} (kT)^2 g'(\mu)$$
Doing Taylors again:
$$\int_{-\infty}^\mu g(\epsilon) \dd{\epsilon} = \int_{-\infty}^{\epsilon_F} g(\epsilon) \dd{\epsilon} + (\mu-\epsilon_F) g(\epsilon_F)$$
$$g'(\mu) = g'(\epsilon_F) + (\mu-\epsilon_F) g''(\epsilon_F)$$

Thus
$$n = n + (\mu-\epsilon_F) g(\epsilon_F) + \frac{\pi^2}{6} (kT)^2 g'(\epsilon_F)$$
and
$$\mu = \epsilon_F - \frac{\pi^2}{6} (kT)^2 \frac{g'(\epsilon_F)}{g(\epsilon_F)} $$

\paragraph{Energy}
\begin{align*}
\expval{E} = \int_{-\infty}^\mu \dd{\epsilon} g(\epsilon) \epsilon + \frac{\pi^2}{6} (k_BT)^2 \qty(g(\mu) + g'(\mu) \mu ) =\\= \int_{-\infty}^{\epsilon_F} \dd{\epsilon} g(\epsilon) \epsilon + (\mu-\epsilon_F) g(\epsilon_F) \epsilon_F + \frac{\pi^2}{6} (k_BT)^2 \frac{g'(\epsilon_F)}{g(\epsilon_F)} g(\epsilon_F)\epsilon_F + \frac{\pi^2}{6} (k_BT)^2 \qty(g(\epsilon_F) + g'(\epsilon_F)\epsilon_F)
\end{align*}
$$ \expval{E} = \expval{E}(T=0) + \frac{\pi^2}{6} (k_BT)^2 g(\epsilon_F)  $$
\paragraph{Quantum numbers of atom}
We know that electron in atom has three quantum numbers $n$,$m$,$l$. But how electrons fill those energies? there are two thumb rules:
\begin{enumerate}
	\item Aufbau principle -- energy shells should be fully filled, from smaller value of $n+l$ to the higher ones.
	\item Madelung's principle -- for equal $n+l$ we choose shell with smaller $n$.
\end{enumerate}

\paragraph{Ionization and electron affinity energies}
The ionization energy is actually can be approximated by summing charge of the nucleus of the atom and all the electrons on the lower shells. Thus the more electrons are on last shell, the higher the energy. Same applies to electron affinity.

\paragraph{Ionic bond}
We want to bond to atoms:
$$Na+Cl \to Na^++Cl^- \to NaCl$$
\paragraph{Electronegativity}
$$\chi = \frac{1}{2} \qty(E_{aff} + E_{ion}) = \frac{1}{2} \qty(E_N - E_{N+1}) + \qty(E_{N-1} - E_N) = \frac{E_{N-1}-E_{N+1}}{2} \approx \pdv{E}{N} \approx -\mu$$

\paragraph{Covalent bonds}
In potential well
$$E=\frac{\hbar^2\pi^2}{2mL^2}$$

If we put two wells close one to another, we get bigger well:
$$\tilde{E} = \frac{\hbar^2 \pi^2}{2\pi(2L)^2}$$

We also have higher energy level in bigger well.
\subsection{Molecular orbitals -- tight-binding model}
\paragraph{Born-Oppenheimer approximation}
Let's write Hamiltonian of two atom nuclei potential:
$$H = K+V_1+V_2$$
For a single electron, $K=\frac{p^2}{2m}$ and
$V_i = \frac{-e^2}{4\pi\epsilon_0\qty(\va{r}-\va{R}_i)}$

We know the solution of
$$H_i = K+V_i$$
denoting eigenfunctions
$$H_i\ket{i} = \epsilon_o \ket{i}$$

We assume that eigenfunction can be approximated
$$\ket{\psi} = \alpha \ket{1} + \beta\ket{2}$$
Also we assume that
$$\braket{m}{n} = \delta_{mn}$$
To find minimum of energy, we need to diagonalize Hamiltonian.  For that we find matrix representation in $\ket{1}$, $\ket{2}$ basis:
$$H_{11} = \mel{1}{H}{1}  = \mel{1}{H_1}{1} + \mel{1}{V_2}{1} = \epsilon_0 + V_{cross}$$
$$H_{21}^* = H_{12} = \mel{1}{H_1}{2} + \mel{1}{V_2}{1} = \epsilon_0 \braket{1}{2}  - t$$
$$H_{22} = \mel{2}{H}{2} = \mel{2}{H_2}{2} + \mel{2}{V_1}{2} = \epsilon_0 + V_{cross}$$
(assuming atoms are identical).

Thus
$$H  =\begin{pmatrix}
\epsilon_0 + V_{cross}  & -t\\
-t^* & \epsilon_0 + V_{cross}
\end{pmatrix}$$

$H$ is of form $H=aI + \va{b} \vdot \va{\sigma}$, i.e., $\epsilon_0\pm = \epsilon_0 + V_{cross} \pm \abs{t}$.

If $t>0$, eigenvectors are $\begin{pmatrix}1\\1\end{pmatrix}$ and $\begin{pmatrix}1\\-1\end{pmatrix}$

If we take in account between nuclei, given that distance between atoms is big, $V_{nn} \approx - V_{cross}$ i.e., $\epsilon_\pm \approx \epsilon_0 \pm \abs{t}$.

If those states are not orthogonal, we can choose some orthonormal basis. Then
$$H\ket{\Psi} = E\ket{\Psi}$$ 
$$\sum_{\beta} \mel{\alpha}{H}{\beta} \braket{\beta}{n}a_n = E \sum_n \braket{\alpha}{n}a_n$$

Define $R_{\beta n} = \braket{\beta}{n}$:
$$H_{\alpha \beta} R_{\beta n}a_n = ER_{\alpha n } a_n$$
$$R^\dagger_{m \alpha} H_{\alpha \beta} R_{\beta n}a_n = ER^\dagger_{m \alpha}R_{\alpha n } a_n$$
Then
$$\mel{m}{H}{n} a_n = ES_{mn} a_n$$
where $S_{mn} = R^\dagger_{m \alpha}R_{\alpha n } = \braket{m}{n}$.
Which is eigenvalue problem:
$$\abs{H-ES} = 0$$
\paragraph{van der Waals force}
Two neutral atoms can become dipole with quantum fluctuations and then create force between them.

\paragraph{Chain of atoms}
Suppose we have a chain of atoms with one orbital per atom and denote $\ket{n}$ the $n^{th}$ atom orbital. We assume $\braket{n}{m} = \delta_{nm}$.

We are doing the same thing, i.e., writing 
$$H=K+\sum_j V_j$$

Assuming $\psi = \sum \alpha_n \ket{n}$ we can rewrite 
$$\sum_{m} H_{nm} \phi_m = E\phi_n$$
$$(K+V_m)\ket{m} = \epsilon_0 \ket{m}$$

Then
$$H_{mn} = \mel{m}{H}{n} = \epsilon_0 \delta_{mn} + \sum_{j\neq m} \mel{m}{V_j}{n}$$
$$\sum_{j\neq m} \mel{m}{V_j}{n} = \begin{cases}
V_0 & m=n\\
-t & m=n+1\\
-t^* & m=n-1\\
0 & \abs{m-n}>1
\end{cases}$$
\end{document}
