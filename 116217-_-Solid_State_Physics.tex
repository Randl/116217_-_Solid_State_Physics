\documentclass[]{article}
\usepackage{amsmath}
\usepackage{amsfonts}
\usepackage{amssymb}
\usepackage{hyperref}
\usepackage{gensymb}
\usepackage{graphicx}
\usepackage{svg}
\usepackage{bbding}
\usepackage{mathtools}
\usepackage{centernot} % not parallel, etc.
\usepackage{lmodern}
\usepackage{morewrites}
\usepackage{xcolor,sectsty} % colorful sections
\usepackage[left=10mm, top=10mm, right=10mm, bottom=20mm, nohead]{geometry}
%\usepackage{bigints}
\usepackage{dsfont} %mathbb 1
\usepackage{esint} % beatiful integrals
\usepackage[arrowdel]{physics}
\usepackage{amsthm} % theorems


\usepackage[T1]{fontenc}
% Nicer default font (+ math font) than Computer Modern for most use cases
% \usepackage{mathpazo} % problems with greek vectors
\usepackage[utf8x]{inputenc} % Allow utf-8 characters in the tex document
% Prevent overflowing lines due to hard-to-break entities
\sloppy 
% Colors for the hyperref package
\definecolor{urlcolor}{rgb}{0,.145,.698}
\definecolor{linkcolor}{rgb}{.71,0.21,0.01}
\definecolor{citecolor}{rgb}{.12,.54,.11}
% Setup hyperref package
\hypersetup{
	breaklinks=true,  % so long urls are correctly broken across lines
	colorlinks=true,
	urlcolor=urlcolor,
	linkcolor=linkcolor,
	citecolor=citecolor,
}


\DeclareFontFamily{OMX}{lmex}{}
\DeclareFontShape{OMX}{lmex}{m}{n}{<-> lmex10}{}


%colors of sections
\definecolor{secfont}{RGB}{46,116,181}
\definecolor{subfont}{RGB}{146,23,57}
\definecolor{parfont}{RGB}{19,127,43}
\definecolor{subparfont}{RGB}{7,11,100}

\subsectionfont{\color{subfont}}
\sectionfont{\color{secfont}}
\paragraphfont{\color{parfont}}
\subparagraphfont{\color{subparfont}}

%\usepackage{babel}[english]
%opening
\title{116217 - Solid State Physics}
\author{Netanel Lindner}
% % Simon, Oxford Solid State PhySics
% Ashcroft and Merim
% Kittel

\newtheorem{theorem}{Theorem}[section]
\newtheorem{definition}{Definition}[section]
\newtheorem{coll}{Collary}[section]
\newtheorem{lemma}{Lemma}[section]

\parindent=0em
\begin{document}


\maketitle

\begin{abstract}

\end{abstract}

%\tableofcontents
\section{Introduction}
Phase are distinguished by symmetries, e.g., liquid and solid have difference in transnational symmetry. Those phase transitions happens together with spontaneous symmetry breakage.For example, magnet ''chooses`` poles when transits to magnetic phase.
\subsection{Heat capacity}
Most of matereals have $C=3k_B$, except diamond in room temperature and pressure. In low temperatures though, materials have $C=\alpha T + \gamma T^3$. Thus, most of materials get to saturation in room temperature, while diamond doesn't.
The linear factor appears only in conductors, thus should come from electrons.
\paragraph{Equipartition theorem}
$$
\expval{x_{m} \frac{\partial H}{\partial x_{n}}}  = \delta_{mn} k_{B} T
$$
in particular, for quadratic degree of freedom we get $\frac{1}{2}k_BT$.

Thus, assuming each atom is harmonic oscillator, we have
$$\expval{E} = 3k_BT$$
and
$$c_V = 3k_B$$
\paragraph{Einstein model}
Now assume the oscillators are quantum:
$$E_n=\hbar \omega\qty(n+\frac{1}{2})$$
The partition function
$$Z = \sum_n e^{-\beta E_n} = \sum_n e^{-\beta \hbar \omega \qty(n+\frac{1}{2})} = e^{-\frac{\beta \hbar \omega}{2}}\frac{1}{1-e^{-\beta \hbar \omega}} = \frac{e^{-\frac{\beta \hbar \omega}{2}}}{e^{-\frac{\beta \hbar \omega}{2}} \qty[e^{\frac{\beta \hbar \omega}{2}}-e^{-\frac{\beta \hbar \omega}{2}}]} = \frac{1}{2\sinh(\frac{\beta \hbar \omega}{2})}$$
$$\expval{E}  = -\pdv{\beta} \ln Z_N = -\frac{1}{Z_N} \pdv{\beta} Z = N \cdot 2\sinh(\frac{\beta \hbar \omega}{2}) \cdot \frac{\hbar \omega}{2}\frac{\cosh(\frac{\beta \hbar \omega}{2})}{\sinh[2](\frac{\beta \hbar \omega}{2})} = \frac{N\hbar \omega}{2} \frac{1}{\tanh(\frac{\beta \hbar \omega}{2})}$$
$$\expval{E}  = N\hbar\omega \qty(n_B(\beta \hbar \omega)+\frac{1}{2})$$
$$n_B(x) = \frac{1}{r^x-1}$$

Now
$$c_V = \pdv{\expval{E}}{T} = 3N k_B (\beta \hbar  \omega)^2 \frac{e^{\beta \hbar \omega}}{\qty(e^{\beta \hbar \omega}-1)^2}$$
$$\lim_{T\to \infty } c_V = 3k_BN$$
$$\lim_{T\to 0 } c_V = 0$$
which is expected, however, we got exponential dependence on $T$ and not $T^3$.
\subsection{Debye model}
Debye proposed to model atoms in solid as sound waves. Plank already quantized EM-waves, the only differences is number of polarization options and speed of wave.


We'll often use periodic bound conditions, since this preserves the symmetry of the system.

Now, for wave $e^{ikx}$ we get $e^{ikx}=e^{ik(x+L)}$, thus we get a condition on wavenumber
$$k=\frac{2\pi}{L}n$$

We also want to switch to integral on $k$.

The connection between frequency and wavenumber is $\omega(\va{k}) = v\cdot \abs{\va{k}}$. The energy per wavenumber
$$\expval{E} = 3\sum_{\va{k}} \hbar \omega(\va{k}) \qty(n_B(\beta\hbar\omega(\va{k}))+\frac{1}{2}) $$
$$\expval{E} = 3\frac{L^3}{(2\pi)^3} \int \dd[3]{k} \hbar \omega(\va{k}) \qty(n_B(\beta\hbar\omega(\va{k}))+\frac{1}{2}) =3 \frac{4\pi L^3}{(2\pi)^3} \int \frac{\omega^2 \dd{\omega}}{v^3} \hbar \omega  \qty(n_B(\beta\hbar\omega)+\frac{1}{2})$$

Now,
$$g(\omega) = L^3 \cdot \frac{12\pi \omega^2}{(2\pi)^3 v^3} = N \frac{12\pi \omega^2}{(2\pi)^3 v^3\rho}$$
is density of states.

Define $\omega_d^3 = 6\pi^2 \rho v^3$, acquiring $g(\omega) = N\frac{9\omega^2}{\omega_d^3} $
$$\expval{E} = \frac{9N\hbar }{\omega_d^3} \int \omega^3 \dd{\omega} \frac{1}{e^{\beta \hbar\omega }-1} + \text{const}$$

Substituting $\beta \hbar \omega = x$:
$$\expval{E} = \frac{9N\hbar }{\omega_d^3 (\beta \hbar)^4} \underbrace{\int \dd{x} x^3 \frac{1}{e^x-1}}_{\frac{\pi^4}{15}} $$

$$\pdv{\expval{E}}{T} = \frac{12\pi^4}{5}\frac{nk_B \qty(k_BT)^3}{(\hbar \omega_d)^3} $$
The reason we get divergence in high temperatures is that we assume any frequency is possible, however, the wavelength can't be less than distance between atoms.

The total number of possible frequencies equals to number of degrees of freedom:
$$\int_0^{\omega_{cutoff}} \dd{\omega} g(\omega) = 3N$$
$$\int_0^{\omega_{cutoff}} \dd{\omega}N \frac{9\omega^2}{\omega_d^3} = N\cdot \frac{3\omega^3}{\omega_d^3}=3N$$
i.e., $\omega_{cutoff} = \omega_d$:
$$\expval{E} = \int_0^{\omega_d} \frac{\dd{\omega} g(\omega) \omega^3}{e^{\beta \hbar \omega}-1}$$

In low temperatures our calculation is yet good, in high temperatures, using Taylor expansion of $e^{\beta \hbar \omega}$
$$\expval{E} = \int_0^{\omega_d} \frac{\dd{\omega} g(\omega) \omega^3}{\beta \hbar \omega} = \int_0^{\omega_d} ]\dd{\omega} g(\omega) k_B T = 3Nk_B T$$

We define some different qunatities based on Debye frequency
$T_d = \frac{\hbar \omega_d}{k_B}$
$k_d = \frac{ \omega_d}{v}$

\subsection{Phonons}
Suppose we have a one-dimensional chain of atoms of mass $m$ in position
$$x_n^{eq} = na$$
with  oscillations
$$u =x_n - x_n^{eq}$$
If $u$ is small it can be approximated with quadratic potential, i.e., a chain of springs.

$$V_{tot} = \sum_j V(x_{j+1}-x_j) = \sum_j \frac{K}{2} \qty(x_{j+1} -x_j -a)^2 = \sum_j \frac{K}{2} (u_{j+1}-u_j)^2 $$
$$F_n = -\pdv{V_{tot}}{u_n} = K(u_{n+1}-u_n) + K(u_{n-1}-u_n)$$

We get system of equations
$$m\ddot{u}_n = F_n = K(u_{n+1} + u_{n-1} - 2u_n)$$
Or, in matrix form
$$m\begin{pmatrix}
\ddot{u}_1\\\ddot{u}_2\\\vdots \\\ddot{u}_n
\end{pmatrix} = K\begin{pmatrix}
-2&1&0&\dots&0\\
1&-2&1&\dots& 0\\
\vdots&\vdots&\vdots&\ddots&\vdots\\
0&0&0&\dots &-2
\end{pmatrix}\begin{pmatrix}
u_1\\u_2\\\vdots \\u_n
\end{pmatrix} $$

Guessing a solution
$$u_n = A e^{i\omega t} e^{-ikan}$$
We get
$$-m\omega^2u_n = K A  e^{i\omega t}\qty( e^{-ika(n+1)}+e^{-ika(n-1)} - 2e^{-ikan} ) = Ku_n(e^{-ika} + e^{ika} -2)$$
$$-m\omega^2 = 2K\qty(\cos(ak)-1)$$
i.e.,
$$m\omega^2 = 4K \sin[2](\frac{ka}{2})$$
or,
$$\omega(k) = 2\sqrt{\frac{K}{m}} \abs{\sin(\frac{ka}{2})}$$
The dispersion has minimum and maximum. We can interval of allowed frequencies ''band`` and its size is called ''bandwidth``. 
The dispersion is periodic with period of $\frac{2\pi}{a}$, since the system is symmetric to translations of $a$.

\paragraph{Solutions of the equations}
Since translation operator commutes with Hamiltonian, the eigenfunctions of Hamiltonian got to be eigenfunctions of the translation operator, which are
$$u_N = A e^{i\omega t - ikna}$$
\subsection{Reciprocal lattice}
\paragraph{Reciprocal space}
Space of wavenumbers. We found equivalent points in inverse space. We call all the points equivalent to $k=0$ reciprocal lattice, i.e., $k=\frac{2\pi}{a} p$ for $p \in \mathbb{Z}$.
\paragraph{First Brillouin zone}
is a unit cell centered in $k=0$.
\paragraph{Finite system}
If the system is finite with size $L$,
$$e^{ik na} = e^{ik \qty(n+\frac{L}{a})a}$$
Thus
$$k=\frac{2\pi}{L} \cdot p$$

i.e., the number of modes
$$N_{modes} = \frac{\frac{2\pi}{a}}{\frac{2\pi}{L}} = \frac{L}{a} = N_{atoms}$$

\paragraph{Properties of dispersion}
\subparagraph{Sound waves}
$$\omega(k) \approx \sqrt{\frac{K}{m}} \abs{k}a = v\abs{k}$$
\subparagraph{Short wavelengths}
$$k \approx \frac{\pi}{a}$$
\subparagraph{Velocity}
$$v_{group} = \dv{\omega}{k}$$
$$v_{phase} = \frac{\omega}{k}$$

\subsection{Quantized phonons }
We quantize our oscillator , with energy
$$E_N =\hbar \omega \qty(n+\frac{1}{2})$$
\paragraph{Crystal momentum}
We define crystal momentum, momentum modulo $\hbar\frac{2\pi}{a}$. Since the lattice can give or take momentum in portions $\hbar\frac{2\pi}{a}$.
\subsection{Two-atom chain of atoms}
Suppose we have two types of atoms, $x$ and $y$, with locations
$$x_n  = na+\frac{a}{10}$$
$$y_n  = na+\frac{8a}{10}$$
Suppose $m_1=m_2$ but $k_1\neq k_2$, ($_2$ connects $x_n$ with $y_n$ and $k_2$ connects $y_n$ with $x_{n+1}$).
$$m\delta \ddot{x}_n = k_2(\delta y_n - \delta x_n) + k_1(\delta y_{n-1} - \delta x_n)$$
$$m\delta \ddot{y}_n =k_1(\delta x_{n+1} - \delta y_{n}) + k_2(\delta x_n - \delta y_n ) $$

Substitute $\delta x_n =A_x e^{i\omega t - ikna}$ and $\delta y_n =A_y e^{i\omega t - ikna}$:
$$\begin{cases}
-\omega^2 A_x m  e^{i\omega t - ikna} = k_2\qty(A_y e^{i\omega t - ikna} - A_x e^{i\omega t - ikna}) + k_1\qty(A_y e^{i\omega t - ik(n-1)a} - A_x e^{i\omega t - ikna})\\-\omega^2 A_y m  e^{i\omega t - ikna}  =k_1\qty(A_x e^{i\omega t - ik(n+1)a} - A_y e^{i\omega t - ikna}) + k_2\qty(A_x e^{i\omega t - ikna} - A_y e^{i\omega t - ikna})
\end{cases} $$

$$\begin{cases}
-\omega^2  mA_x = (e^{ika} k_1 + k_2)A_y  - (k_1+k_2)A_x \\
-\omega^2  m A_y   =\qty(k_1 e^{ - ika} +k_2) A_x -  (k_1+k_2)A_y
\end{cases} $$

In matrix form
$$m\omega^2 \begin{pmatrix}A_x\\A_y\end{pmatrix} = \begin{pmatrix}
- (k_1+k_2)&e^{ika} k_1 + k_2\\
k_1 e^{ - ika} +k_2 & -  (k_1+k_2)
\end{pmatrix} \begin{pmatrix}A_x\\A_y\end{pmatrix} $$
$$\begin{pmatrix}
m\omega^2 - (k_1+k_2)&-\qty(e^{ika} k_1 + k_2)\\
-\qty(k_1 e^{ - ika} +k_2 )& m\omega^2 -  (k_1+k_2)
\end{pmatrix} \begin{pmatrix}A_x\\A_y\end{pmatrix} $$

$$\qty[m\omega^2 - (k_1+k_2)]^2 + \abs{k_1 e^{ - ika} +k_2}^2 = 0$$
$$m\omega^2 = k_1+k_2 \pm \abs{k_1e^{ika} + k_2}$$
$$\omega = \sqrt{\frac{k_1+k_2}{m} \pm \frac{1}{m}\sqrt{k_1^2 + k_2^2+2k_1k_2\cos(ka)}} = \sqrt{\frac{k_1+k_2}{m} \pm \frac{1}{m}\sqrt{(k_1 + k_2)^2-4k_1k_2\sin[2](\frac{ka}{2})}}$$

Take a look at eigenvectors in $k=0$:
they are $\begin{pmatrix}
1\\1
\end{pmatrix}$ for eigenvalue $0$ and $\begin{pmatrix}
1\\-1
\end{pmatrix}$ for eigenvalue $2(k_1+k_2)$. 

\input{lect5.tex}
\end{document}
